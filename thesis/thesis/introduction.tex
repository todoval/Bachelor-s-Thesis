\chapter{Introduction}

Over the last few years, the importance and usage of digital documents has grown rapidly. Today, almost everything can be found in digital form. However, there are still tons of documents that need to undergo the process of digitization - conversion of analog signals or information into a digital format. Creating a digital document for viewing from an analog input is not hard - a simple scan should suffice. Complications arrive when the user wishes to modify the document. What a scan provides is merely an image, which has no information about its elements - text, paragraphs, numberings, headers/footers, forms, tables... This is the moment when an OCR software is needed.

Optical Character Recognition is a widespread technology used for converting images containing written text into a machine-readable text data. It became popular during early 1990s while trying to digitize historic newspaper, and is now frequently used for data entry automation, ticket and voucher validation, tax-free shopping, automatic number plate recognition, as well as assisting blind and visually impaired people.

Different OCR softwares use different techniques to achieve their goal. They mostly consist of a variety of combined heuristic algorithms with an equally important part of preprocessing. Although most of these softwares guarantee their accuracy rate to be above 98\%, the measurement techniques are not always the same and therefore the rates can be misleading.
Furthermore, most of these softwares already work on different kinds of assumptions about documents - that they must have a specific layout, contain a header and a footer, have all text lines horizontally aligned et cetera...

In this thesis we specifically focus on recognition of tabular data and the problems that come along with it. We describe the possibilities and 



\paragraph{Related Work}


// co je digitalizacia, je popularna, potrebujeme zistit viac ako len obrazok,
treba informaciu o dokumente
// co robia OCR softwares
// ake su problemy s OCR softwares
// this thesis is concerned by.... riesi tabulky. Momentalne su taketo problemy s detekciou tabuliek atd atd
// this thesis ma nasledovnu strukturu
Related work



1.3 Goal
Many software projects have already been implemented, trying to solve this
problem, or at least a part of it - line detection, image detection, logical text
detection. There have already been a few attempts at solving table detection.
However, mostly they can only select the whole table - not its header, lines,

1

elements and logical connections between them. Therefore, exporting the
table for future editing, for example to Microsoft Excel, is out of the picture.
2 Problems
Following are a few most common issues and also reasons why none of the
software applications implemented for this are ever perfect.
2.1 Logical succession

In a document (mostly in newspapers, magazines, posters...), logical suc-
cession of text paragraphs can be a problem for a regular reader (which

text belongs to which image, which paragraphs belongs to which title, etc.),
nonetheless for a software. Therefore, heuristics, or even artificial intelligence
is used to solve this. This is not such a big issue with tables, however, even a
few elements can be colored differently and completely change the meaning
of the table.
2.2 Weak detection of lines
Upon scanning a document, lines can get disrupted or even deleted. Possible
solution for this are convolution based techniques using Hough Transform,
edge detectors (Canny, Sobel, Compass, Zero Crossing, Roberts Cross...) or
other graphics techniques.
2.3 Non-horizontal and non-vertical lines
Sometimes, tables are not parallel to page edges. Simple approaches, such
as reading of pages by lines from the top, become unusable.
3 Program design
3.1 User Interface
The UI will be implemented first. It will include a few user options - loading
options for existing documents that need to be detected, saving and exporting

2

options once tables are detected. It will also contain a run button, starting
the actual program for detection.
3.2 Backend
Backend will be the core of the program. It will contain the implemented
table detection algorithm along with a few possible features. There will most

likely be a lot of graphics computations (transformations, gradient calcula-
tions) used for the OCR (Optical Character Recognition).

4 Implementation schedule
4.1 Design and communication
Design of the program, user interface and communication between UI and

backend. Important classes, functions and their overview. Research of spe-
cific algorithms.

Expected November 2018.
4.2 UI
User Interface implementation with all user options, as well as importing
already scanned PDF documents. Implementation of communication with
backend.
Expected January 2019.
4.3 Table Detection
The most complicated part, using heuristics. Output in the form of .xsl (or
other table format) file, or at least a structure implementing a table.
Expected March 2019.
4.4 Features

Features like exporting tables to table formats for future editing and pro-
cessing, whether for MS Excel or other table processing applications. Also,

a communication with scanner might be useful.
Expected May 2019.