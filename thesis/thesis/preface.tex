\chapter*{Introduction}
\addcontentsline{toc}{chapter}{Introduction}

Over the last few years, the importance and usage of digital documents has grown rapidly. Today, almost everything can be found in digital form. However, there are still tons of documents that need to undergo the process of digitization - conversion of analog signals or information into a digital format. Creating a digital document for viewing from an analog input is not hard - a simple scan should suffice. Complications arrive when the user wishes to modify the document. What a scan provides is merely an image, which has no information about its elements --- text\todo{Existujou 3 pomlcky: kratkej hyphen na slepovani vice-slovnych-vyrazu, delsi spojka (en-dash) na veci jako `az', tj. 10--20, a pomlka (em-dash) --- coz je to cos tady chtela pouzit}, paragraphs, numberings, headers/footers\todo{lomitka v textu nejsou moc OK}, forms, tables, \dots \todo{vsechny ellipsis se formatujou jako dots \dots nebo odpovidajicima makrama} This is the area where OCR software is needed.

Optical Character Recognition is a widespread technology used for converting images containing written text into a machine-readable text data. It became popular during early 1990s while trying to digitize historic newspaper, and is now frequently used for data entry automation, ticket and voucher validation, tax-free shopping, automatic number plate recognition, as well as assisting blind and visually impaired people.

Different OCR softwares use different techniques to achieve their goal. They mostly consist of a variety of combined heuristic algorithms with an equally important part of preprocessing. Although most of these softwares guarantee \todo{guarantees vs reaches?} their accuracy rate to be above 98\%, the measurement techniques are not always the same and therefore the rates can be misleading.

Furthermore, most of these softwares already work on different kinds of assumptions about documents --- that they must have a specific layout, contain a header and a footer, have all text lines horizontally aligned, etc.

In this thesis we specifically focus on recognition of tabular data and the problems that come along with it. We describe the possibilities and 



// co je digitalizacia, je popularna, potrebujeme zistit viac ako len obrazok, treba informaciu o dokumente


// co robia OCR softwares


// ake su problemy s OCR softwares


// my konkretne riesime tabulky


// thesis bude mat nasledovnu strukturu


\paragraph{Related Work}
