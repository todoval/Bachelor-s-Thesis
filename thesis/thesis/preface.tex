\chapter*{Introduction}
\addcontentsline{toc}{chapter}{Introduction}

Recently, digitalization of documents has become an important part of information systems and related workflows. By digitalization, documents from governmental, administrative, educational and publishing workflows are being processed to a more accessible, searchable and manageable form.

To convert the printed text documents back into digital text form, digitalization tools employ \emph{optical character recognition} (OCR). OCR engines are used to recognize individual text elements of a scanned document and output them in a more suitable format, e.g.~as rich text or a searchable PDF file.

Specialized OCR algorithms exist to handle various specific input document layouts where general OCR would produce sub-optimal or unusuable results; e.g. for tickets, passports, car plates or postal envelopes. Tabular documents form a wide, useful category of structured input --- table recognition algorithms must be able to process various features found in the tables, including inconsistent border formatting, different alignment, spanning cells, or misaligned columns and rows. Currently, the available table-recognition software is, in most cases derived, from text-oriented OCR algorithms and recognizes only limited amount of all possible table features.

The main goal of this thesis is to explore the existing OCR techniques and software, and create a new table-recognition OCR software that can provide better results than the currently available tools. The resulting software uses an external OCR engine and image-processing libraries to preprocess the input images and extract simple text elements, which are then heuristically combined into words, lines, table columns and whole tables. Finally, it produces an easily readable JSON-formatted output that describes the layouts of tabular and textual elements in the processed document. The generalized JSON output may eventually converted to other, more specific formats.

\subsection*{Related work}

There exist several softwares that focus on table recognition. For example, Tabula~\cite{Tabula} is an under development open-source project focusing on the extraction of tables from PDF files. This software uses the help of the OpenCV library~\cite{OpenCV} and often results in errors when presented with scanned documents, multi-line rows and complex tables. Another examples are OCRSpace~\cite{OCRSpace} and OpenCV~\cite{OpenCV}. Although the focus of these engines is mainly character recognition, they also provide table recognition of specific layouts (e.g. receipts, tickets). Other approaches to table recognition were presented by~\citet{pdf2table},~\citet{TRecs} and~\citet{MediumTable}. Even though they proposed new techniques, they are now either outdated or do not provide results of sufficient quality. Satisfactory results are provided by commercial softwares like ABBYY FineReader~\cite{ABBYYdpi}.

Table recognition is also implemented as a feature of the Tesseract library~\cite{tableDetHeterogeneous}. We \xxx{will}\todo{vygrepuj si z toho vsechny `will' a odstran je. Will je budoucnost, ten text a diskuze tam uz ale je.} discuss it in detail later in this thesis.\todo{tady primej ref na sekci}

\paragraph{Layout of this thesis} This thesis is structured as follows: In Chapter 1, we review the obstacles that complicate character recognition and show existing solutions for some of them; after that we describe several techniques and heuristics for text recognition. In Chapter 2, we describe implementation of existing table recognition algorithms. Chapter 3 details the implementation of our software, including the heuristic used for table detection and recognition. Performance measurements, results and summary of the improvements are presented in Chapter 4. After the last chapter, we conclude with a brief overview of possible future work.
