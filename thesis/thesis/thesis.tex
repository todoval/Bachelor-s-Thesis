%%% The main file. It contains definitions of basic parameters and includes all other parts.

%% Settings for single-side (simplex) printing
% Margins: left 40mm, right 25mm, top and bottom 25mm
% (but beware, LaTeX adds 1in implicitly)
%\documentclass[12pt,a4paper]{report}
%\setlength\textwidth{145mm}
%\setlength\textheight{247mm}
%\setlength\oddsidemargin{15mm}
%\setlength\evensidemargin{15mm}
%\setlength\topmargin{0mm}
%\setlength\headsep{0mm}
%\setlength\headheight{0mm}
% \openright makes the following text appear on a right-hand page
%\let\openright=\clearpage

%% Settings for two-sided (duplex) printing
\documentclass[12pt,a4paper,twoside,openright]{report}
% \setlength\textwidth{145mm}
% \setlength\textheight{247mm}
% \setlength\oddsidemargin{14.2mm}
% \setlength\evensidemargin{0mm}
% \setlength\topmargin{0mm}
% \setlength\headsep{0mm}
% \setlength\headheight{0mm}
\let\openright=\cleardoublepage

%% Generate PDF/A-2u
\usepackage[a-2u]{pdfx}
\usepackage{algorithm}
\usepackage{comment}
\usepackage{algpseudocode}

%% Character encoding: usually latin2, cp1250 or utf8:
\usepackage[utf8]{inputenc}

% Let's use a decent modern font
\usepackage[T1]{fontenc}
\usepackage[mono=false]{libertine}
%other choices -- URW schoolbook: 
%\usepackage{fouriernc}
% or URW palladio
%\usepackage[sc]{mathpazo}
%\linespread{1.05}

%% Further useful packages (included in most LaTeX distributions)
\usepackage{amsmath}        % extensions for typesetting of math
\usepackage{amsfonts}       % math fonts
\usepackage{amsthm}         % theorems, definitions, etc.
\usepackage{bbding}         % various symbols (squares, asterisks, scissors, ...)
\usepackage{graphbox}       % figure alignment
\usepackage{bm}             % boldface symbols (\bm)
\usepackage{graphicx}       % embedding of pictures
\usepackage{fancyvrb}       % improved verbatim environment
\usepackage[style=numeric,natbib=true,backend=bibtex,style=alphabetic]{biblatex}
\addbibresource{bibliography.bib}
\defbibheading{Bibliography}{}

\usepackage[nottoc]{tocbibind} % makes sure that bibliography and the lists
			    % of figures/tables are included in the table
			    % of contents
\usepackage{dcolumn}        % improved alignment of table columns
\usepackage{booktabs}       % improved horizontal lines in tables
\usepackage{paralist}       % improved enumerate and itemize
%\usepackage[usenames]{xcolor}  % typesetting in color
\usepackage[textsize=tiny, color=yellow!30]{todonotes}
\usepackage{longtable}
\usepackage{subcaption}
\usepackage{listings}
\usepackage{cleveref}

\newcommand{\xxx}[1]{\textcolor{red}{#1}}

%%% Basic information on the thesis

% Thesis title in English (exactly as in the formal assignment)
\def\ThesisTitle{OCR for tabular data}

% Author of the thesis
\def\ThesisAuthor{Lucia Tódová}

% Year when the thesis is submitted
\def\YearSubmitted{2019}

% Name of the department or institute, where the work was officially assigned
% (according to the Organizational Structure of MFF UK in English,
% or a full name of a department outside MFF)
\def\Department{Department of Software Engineering}

% Is it a department (katedra), or an institute (ústav)?
\def\DeptType{Department}

% Thesis supervisor: name, surname and titles
\def\Supervisor{Mgr. Miroslav Kratochvíl}

% Supervisor's department (again according to Organizational structure of MFF)
\def\SupervisorsDepartment{Department of Software Engineering}

% Study programme and specialization
\def\StudyProgramme{Computer Science}
\def\StudyBranch{Programming and Software Systems}

% An optional dedication: you can thank whomever you wish (your supervisor,
% consultant, a person who lent the software, etc.)
\def\Dedication{%
Dedication.
}

% Abstract (recommended length around 80-200 words; this is not a copy of your thesis assignment!)
\def\Abstract{%
Abstract.
}

% 3 to 5 keywords (recommended), each enclosed in curly braces
\def\Keywords{%
{key} {words}
}

%% The hyperref package for clickable links in PDF and also for storing
%% metadata to PDF (including the table of contents).
%% Most settings are pre-set by the pdfx package.
\hypersetup{unicode}
\hypersetup{breaklinks=true}

% Definitions of macros (see description inside)
\include{macros}

% Title page and various mandatory informational pages
\begin{document}
\include{title}

%%% A page with automatically generated table of contents of the bachelor thesis

\tableofcontents

%%% Each chapter is kept in a separate file
\chapter*{Introduction}
\addcontentsline{toc}{chapter}{Introduction}

Digitalization of documents has become a demand in this century. All kinds of administrative, school and press documents are being processed to a scalable, more accessible digital form. 

To achieve this, we use a so-called \emph{optical character recognition} (OCR). OCR engines are used to recognize individual elements of a scanned document and output them in a desired format, e.g. text, searchable PDF, etc.

OCR algorithms oriented on specific input document layouts (for example tickets, passports, car plates) work sufficiently. A more generically focused OCR engines tend to have problems with complex layout structures, which might be difficult to comprehend even for a human perception. One of these structures are tables.

Table recognition algorithms are hard to generalize due to various aspects, such as diverse or even missing borders, different cell sizes, multi-column and multi-row cells, element alignments, etc. 

The main goal of this thesis is to research possible OCR techniques and apply this knowledge to recognize table elements on top of character recognition. Additional aim is to create a portable table structure (JSON).

Upon meeting specific input conditions, the resulting JSON structures provide a convenient overview of the tables inside a document in many of the tested cases. Better results would be achieved with more accurate character recognition provided by external open-source software used by this thesis (Tesseract~\cite{TesseractGIT}).

\subsection*{Related work}

There exist a lot of other OCR engines which provide similar features as Tesseract. Under existing open-source OCR software are, for example, CuneiForm, GOCR, Ocrad or OCRopus. They all support basic character recognition, with slight differences in the number of recognized languages, types of symbols, provided features, accuracy of results and in the general focus of the recognition (e.g. scanned documents, barcode detection). Although they all produce satisfactory results, none of them provides the user with table recognition.

Table recognition is a problem addressed by algorithms presented by~\citet{tableDetHeterogeneous}, \citet{TRecs}, \citet{MediumTable}, \citet{pdf2table} and many more. However, most of the implementations are either outdated, are not open-source lincensed, or do not provide sufficient results.

\paragraph{Layout of this thesis} This thesis is structured as follows: In Chapter 1, we review the obstacles that complicate character recognition and present ways of their elimination. Moreover, we present techniques of heuristic text recognition. In Chapter 2, we describe the individual steps of existing table recognition algorithms. Chapter 3 describes the implementation of our software, including our heuristic algorithm used for table detection and recognition. Performance measurements, results and possible implementation improvements are presented in Chapter 4. After the last chapter, we conclude with a brief overview of possible future work.


\chapter{Text Recognition Algorithms}

The root of any complex OCR software is text recognition.
Although this is easily done in printed documents, there are a lot of factors that complicate this process.
Based on different OCR documentations(...zdroje?), the few of the most important problems were found to be:
To name a few of the most important:
\begin{itemize}
\item\texttt{different fonts} - Nowadays, there exist a number of fonts and styles. To hold the information about each and every one would be unsustainable for any OCR software. Therefore, it needs to make assumptions and use heuristics to match its vision of the character with the one it is reading.
\item\texttt {handwritten documents} - The software has no notion about whether the document was handwritten or not. It sees the characters as merely a more complicated font. On contrary to the font, however, the characters that are supposed to be the same do not always have to look that way, .... (because of people/given the mistakes of people).  This is why the OCR software has to count on (???) minor mistakes the .... odchylka depends on different factors - line height, length, character size... Furthermore, handwriting can sometimes be unrecognizable even by a human eye. This case often prompts the software to fail.
\item\texttt {low quality of the scanned document} - This can be caused by a poor scan or just a poor initial image and mean a variety of things. The image can be low-contrasted, not sharp enough, lines can be disrupted or pixelated (mostly in case of a low scan DPI), it can contain a lot of noise... These problems are usually (partly) solved during the preprocessing part of OCR algorithm.
\item\texttt {skew problems} - These problems also fall under the part of preprocessing. As the result of a scan or even a photography is almost never an image with a perfectly aligned text as it was on the paper, this needs to fixed, so the actual algorithm can count on correctly aligned images.
\item\texttt {colors} - The general rule is - the better the contrast of the image, the better the OCR results. Colored images generally have a lower contrast than when they are in black and white. Before passing to OCR, they therefore undergo binarization.
\item\texttt {similar characters} - Even people sometimes make the mistakes when distinguishing characters like S and 5, O and 0, or I and l. OCR software can often make the same mistake, especially when different, special fonts are used, or in case of handling handwriting.
\item\texttt {inconsistent font use} - To determine word spacing, most OCR engines calculate gaps between characters and heuristically decide which characters belong together. With different fonts, spaces also differ, which may produce unsatisfactory results.
\item\texttt {...} noise?
\end{itemize}

These and multiple other factors are also the reason why many OCR softwares work on "assumptions" about documents - that the document has only one column, is not handwritten, that its lines are horizontally aligned et cetera. Also, this is why no OCR software can work perfectly and will always keep encountering documents that are "unrecoverable". 

However, there are many ways in which we can improve the results of the OCR.

\section{Preprocessing}

OCR algorithms usually do not work well on most images provided by user. It is because they(???) are either malformed, disrupted or have other of the problems mentioned above.

Most of the OCR engines already come with some kind of built-in preprocessing filter. The problem with those built-in filters is that they most likely do not match your case and are usually very simple.

This is why the best practice is to firstly preprocess the image and then pass the result to the OCR algorithm.

In this section, we will discuss the most important image transformation that we can perform to improve the result of the OCR. 

// existuju aj ine transformacie, ako sharpening, alpha correction etc, add small border


\subsection{Scaling}

Usually in case of OCR, more is better. The more information the OCR engine can use to interpret text, the more accurate the result will be. This can be specifically applied to the case of resolution and bit-depth.

(zdroje?)
Almost all OCR engines(Tesseract, OpenCV, ...zdroje?) encourage their users to use a 300 DPI images. Their reason for that is pretty simple - it is the point where you gain the most accuracy without sacrificing speed and file size. For example, try running a few tests for the same image with different resolution on OCR engines. (after running a few tests? zdroje?) You will see that the improvement gap between 200 DPI scan and 300 DPI scan will be at least 2 times the improvement gap of any other resolutions. Also, comparing other images above 300 DPI with a 300 DPI image, you find the improvement gap to be nearly absent. It is obviously still there, but in almost all cases, the higher DPI is not worth the time and space. 

Another reason for the use of 300 DPI is the fact that most OCR engines are trained to this resolution. Some engines (to do - ktore?), no matter what resolution you give them will actually sample up or down to get to 300 DPI. 

Achieving a 300 DPI by a scan is simple enough. If the user has already provided us with an image of a different resolution, our next steps depend on what the resolution is:
\begin{itemize}
\item\texttt {DPI of new image < 300 :} If we were to just scale the image, the result would be pixelated, blurred and not sharp enough (unsharp???). The most affected parts would be the diagonal edges of elements. They would appear jagged (\emph{aliased}). ( add a sample picture somewhere over here !!!). (This gives the image overall a poor quality?). To minimize this unwanted effect, a technique called \emph{interpolation} is used.

Interpolation works by using known data to estimate values at unknown points. It specifically approximates the resulting pixel's color and intensity based on the values at surrounding pixels. It therefore already includes the process of \emph{anti-aliasing} (process used to minimize aliases), as it is based on the same technique.

The more data we have, the better the interpolation - therefore we will still see the difference between a resized image from 72 DPI to 300 DPI and 200 DPI to 300 DPI.

Interpolation algorithms can be grouped into two categories: \emph{adaptive} and \emph{non-adaptive}. Non-adaptive methods treat all pixels equally, while adaptive methods change depending on what they are interpolating, specifically smooth texture vs. sharp edges. Adaptive methods are primarily designed to minimize the presence of interpolation artifacts in regions where they are most apparent, and they differ by the way they detect edges.

(adaptive https://www.researchgate.net/publication/224647180_Adaptive_Interpolation_Algorithm_for_Real-time_Image_Resizing)

Non-adaptive methods do not distinct different pixels. Their complexity depends on the number of adjacent pixels during interpolation, which is also the criterion by which the existing methods are divided. 

\begin{itemize}
\item\texttt {Nearest Neighbor Interpolation: } This algorithm considers only one pixel - the closest one to the interpolated point.

\item\texttt {Bilinear Interpolation } considers the closest 2x2 neighborhood of known pixel values surrounding the unknown pixel. It then assigns the weighted average of these 4 pixels to the final interpolated value.

\item\texttt {Bicubic Interpolation } goes one step beyond bilinear and takes the closest 4x4 neighbours, which sums up to the total of 16 pixels. During the calculation of the final value, pixels closer to the interpolated point are given a higher weighting.

\end{itemize}

( -> https://www.researchgate.net/publication/301889708_Image_Interpolation_Techniques_in_Digital_Image_Processing_An_Overview)

There also exist higher order interpolations, such as \emph{Spline}, \emph{Sinc}, \emph{Lanczos}... They take more surrounding pixels into consideration and calculate the resulting value through (more complicated functions? to-do - pochopit ako to presne robia, netreba ak dam referenciu?). The results are better than just simple calculation, but they take up way more time. During OCR, such complex and time-consuming functions are not necessary, as we are not yet working with any rotations and distortions and just need to resize the existing image.

For the best quality/time ratio, the popular decision in most cases (zase spomenut tesseract, opencv etc?) is \emph{Bicubic Interpolation}. Although \emph{Neareast Neighbor} and \emph{Bilinear} methods are extremely fast, the results were found to be poor.

The more pixels, the more accurate the interpolation result. This obviously comes at the expense of a processing time.

\item\texttt {DPI of new image > 300 : }Although we are solving the exact opposite problem as in previous case, the approaches to this are quite similar. In this case, though, we need to decrease the number of pixels and decide what will the values of the new ones be. The easiest approach would be to "pick every nth pixel", but the same problems as previously arise - blurring, aliasing, pixelation...

The easiest and most widely used approaches are very similar to what is already explained above. To calculate the value of the resulting pixel, we choose a number of surrounding pixels (as in previous case, the 4x4 turns out to be working ... best). We calculate their weighted average and assign this value to the unknown pixel.

There are obviously many other ways that our desired result can be achieved. Already mentioned \emph{Spline}, \emph{Sinc}, \emph{Lanczos} and other "reverse interpolations" are a few of them. Worth noting are also \emph{Fourier transformation}, \emph{perceptual based methods},\emph{content-adaptive methods} and other adaptive and heuristic techniques.
Many of these methods produce even better results, but, as mentioned before, most preprocessing engines still stick to the simpler "reversed Bicubic Interpolation" in a successful attempt to speed up the algorithm.

\end{itemize}

\subsection{Deskew}

Skewed images are probably the most common problem that many OCR engines struggle with. They usually .. (arise? happen?) when an image is scanned or photographed crookedly. 

One of the steps of OCR algorithms is \emph{page segmentation}. It is the process that decomposes a document image into elements. This division works with (based on?) vertically and horizontally aligned characters, spaces and other elements. If the documents have Manhattan layout, decomposition is even easier and can be done by vertical and horizontal cuts. We will discuss this topic thoroughly in the following chapter. (For now, what we need to know is that?) skewed images pass misaligned data to the OCR algorithm. This causes more errors and ultimately leads to poorer results.

There are different approaches that deal with this problem. All of these methods work after binarization of the image (for more accurate results) and generally assume the skew angle to be no more than 15$^{\circ}$. As mentioned in (ref), algorithms that work on greater skew angles also exist. They are not as widely used, though, as most of the time, correction needs to be done on scanned documents - which are mostly only slightly tweaked.

To name a few of the most promising algorithms:

\begin{itemize}

\item\texttt{Hough Transform:} As described in (hugh ref), this method is used to find straight lines in image and given their rotation, determine the skew angle of the whole documents. It works by saving results of Hugh Transform on different lines, and comparing which line has the most pixels. That gives us the line that is most represented in the input image, which gives us the resulting angle. Although highly effective, computations of transform can be quite time consuming. This is why Hugh Transform usually is not done on every point of the image, but only a set of chosen input points. Another problem occurs if pictures are present in the input image (as Hugh Transform counts with aligned pixels). To solve this, we can not choose input points from the given picture.

\item\texttt{Projection Profile: } Projection Profile method is based on creating a histogram of black pixels along horizontal or vertical scan lines. For an image with horizontal scan lines, this histogram will have peaks at text line positions. The algorithm estimates the skew by rotating the image in different angles and comparing the resulting histograms. The one with the most peaks and valleys belongs to our final angle. The problem with this technique is that it works on text areas only, as images or other elements can disrupt the histograms.

\item\texttt{Cross Corelation} works similarly to Projection Profile method mentioned above. Firstly, it selects a number of vertical segments from the image. It then performs the Projection Profile algorithm on these segments and by comparing then, it can estimate the final skew angle.
The drawbacks of this method are similar to those of Projection Profile. However, it usually needs to process less data with no loss in accuracy, given a good choice for the width of the columns.

\item\texttt{Connected Component Clustering} - or  \emph{Nearest Neighbor Clustering}. Thoroughly described in (ccref), these algorithms are based on finding coherency between connected components of the image. They either compute the skew of every component (using their bottom line) and compare the results, or group components by neighbours and create a group histogram for examination, or many other more. (maybe drawbacks? As the algorithms greatly differ, each has its own drawbacks. cc vzdy rovnaky algo)

\end{itemize}

Worth noting are also other methods like \emph{Fourier}, \emph{Wavelet Decomposition}, \emph{Radon Transform}, \emph{PCP}, \emph{one-pass} or \emph{multi-pass} skew correction, \emph{morphology algorithms} etc... These are all briefly described in (ref).

(talk more about fourier?)

- fourier - http://www.ijsce.org/wp-content/uploads/papers/v3i4/D1739093413.pdf
- ccref - http://citeseerx.ist.psu.edu/viewdoc/download?doi=10.1.1.121.3963&rep=rep1&type=pdf
- hugh ref - https://pdfs.semanticscholar.org/a995/14ebd9caf97ec8fa80da58bcb0f9b2fb93e1.pdf
- one pass skew - https://ac.els-cdn.com/S1319157803800034/1-s2.0-S1319157803800034-main.pdf?_tid=7f3311c1-8a8f-48be-888d-7a320f491dce&acdnat=1543157733_ff26c35ff333bd3388cb412e4522f0d7
- referencia - https://www.researchgate.net/publication/263540666_Image_Skew_Detection_A_Comprehensive_Study

\subsection{Contrast enhancement}

A \emph{histogram} is an accurate representation of the distribution of data. In case of image histograms, it represents the tonal distribution of the image - x-asis in histogram stands for all available tonal levels, and y-axis represents the number of pixels for each tonal level. (mby picture?)

Grayscale images are represented by only one histogram, where the x-axis contains all the available grey values. However, in colored (RGB) images, the histogram is displayed by the terms of three separate histograms - one for each color component (R, G and B). Increasing contrast for colored pictures is therefore a more complex task. This, however, is not a case we need to solve, as the next part of preprocessing is binarization. As we will mention later, binarization works better on grayscale images. Therefore, the conversion to grayscale can already be performed in this section. Only after that, the contrast enhancement will take place.

\texttt{Grayscale conversion} is the process of computing and assigning the grey value for each colored pixel from its attributes. Over the years, different approaches have been created. That is why, when using image editing software such as Adobe Photoshop or GIMP, there are many options available for grayscale conversion, each of them producing slightly different results. 

The most simple approach is to simply find the average of the R, G and B values. This does not really preserve the brightness of the image. Therefore, corrections like luma are used. This algorithm is based on the fact that humans perceive different colors differently (green more strongly than red, and red more strongly than blue), therefore, each color is weighted dependant on how it is perceived by the human eye. Another way to approach grayscale converstion is to represents the color based on its HSL values and convert it to its least saturated variant.

As decribed in (zdroj other grayscale more), these exist many more algorithms approaching this problem. Most of them require more complicated computations and try to preserve attributes of the image that may be lost during the process - like contrast or luminance.

Now that we have a grayscale image, the next step is \emph{contrast enhancement} with the help of already mentioned histograms. In them, contrast is represented as distribution of pixels. The more evenly the pixel counts cover a broad range of grayscale levels, the better the contrast of the image.  Pixel counts that are restricted to a smaller range indicate low contrast.

To achieve an image with a higher contrast, we therefore need to "stretch" its histogram. Following are a few methods that approach this problem. Other types of methods can be found (refs)

\begin{itemize}

\item\texttt{Linear Stretching } method linearly expands the original digital values of the histogram into a new distribution. As mentioned in (ref), there exist three methods of linear stretching - \emph{Min-Max}, \emph{Percentage} and \emph{Piecewise} Contrast Stretch. All these methods make subtle variations within the image data more obvious, are are best applied to remotely
sensed images with Gaussian or near-Gaussian histograms.

\item\texttt{Non-linear Stretching } is done by functions with non-linear form. These include, for example, \emph{logarithmic transform}, \emph{exponential transform} or \emph{Power Law} mentioned in (ref). The downside of these methods is that each value in the input image can have several values in the output image - therefore, the objects in the original scene lose their correct relative brightness value.

\item\texttt{Histogram Equalization} is one of the most popular methods of non-linear stretching. As described in (ref), this method is based on transforms dependant on the probability distribution function of the histogram. Although it produces unrealistic effects in photographs, it is widely used for scientific images.

\end{itemize}

- logarithmic and power law - http://academica-e.unavarra.es/bitstream/handle/2454/2368/577327.pdf?sequence=1
- linear stretch: http://www.academia.edu/9301934/Comparative_Study_of_Linear_and_Non-linear_Contrast_Enhancement_Techniques
- histogram eq: https://www.researchgate.net/publication/283727396_Image_enhancement_by_Histogram_equalization
- simple grayscale: https://journals.plos.org/plosone/article?id=10.1371/journal.pone.0029740
- other grayscale more: http://cadik.posvete.cz/color_to_gray_evaluation/cadik08perceptualEvaluation.pdf
- other grayscale: http://citeseerx.ist.psu.edu/viewdoc/download?doi=10.1.1.221.4952&rep=rep1&type=pdf
- other enhance - https://arxiv.org/ftp/arxiv/papers/1003/1003.4053.pdf
- other ehance - https://www.sciencepubco.com/index.php/ijet/article/view/14811/6062

\subsection{Binarization}

As already mentioned, element detection in color, or even greyscale images is a way more complicated task than with monochrome one (1 bit) images. This is due to the the lack of contrast and therefore possible blending of different edges during recognition, as mentioned in ref (zdroj).

This is when binarization comes in handy. Binarization is a process that converts an image to its monochrome one form. This means that every pixel on the binarized image will be either black or white.

Many binarization algorithms work under the assumption that the input image is already in greyscale. If not, they firstly transform the image to greyscale, then apply the binarization techniques. This does not concern us, though. In the last section, where we talked about contrast enhancement of the image, our results were already in greyscale. Therefore, for the next part, let us assume that the images we work with are strictly in greyscale.

As mentioned in (zdroj global), the simplest approach to binarization is \emph{global thresholding}. It works on choosing a threshold value and iterating over the whole image, comparing the value of each pixel to this threshold. If it is greater, the pixel in resulting image will be black, otherwise white. As the brightness and contrast of different images vary, it is impossible to choose a threshold suitable for all images.

Therefore, other more complex methods are used:

\begin{itemize}
\item\texttt{Otsu's method} is one of the most widely used and mentioned methods for image binarization. First, it computes the threshold value from the input image and then applies global thresholding. To get the optimal value of the threshold, Otsu's method works by dividing the pixels into two groups - background and foreground pixels. It chooses these groups so that the within-group variance is minimized and between-group variance is maximized. Upside to this method is that it is simple and easy to implement. However, upon applying it to images with uneven dark or bright spots, it (shows/deems to be?) unusable.

\item\texttt{Local Thresholding: } In contrast to Otsu's method, where all pixels behaved regarding to their common threshold value, local thresholding methods assign different segments of the image different threshold value. This solves the problem with unevenly lit spots - darker spots will be assigned a lower threshold value, while brighter will be assigned a higher threshold value. The difference between different local thresholding algorithms is the way the values are assigned (somewhere here reference thresholding?). For example, \emph{Niblack's Method} determines the threshold of every pixel by using the mean and standard deviation of surrounding pixels, which causes errors in blank regions of the image. \emph{Sauvola Method} improves this by using the dynamic range of image gray-value standard deviation. However, this often results in text thinning. Another method worth mentioning is \emph{Bernsen's}, where the threshold is set to the mean value between the lowest and highest gray-level pixel in the neighborhood.

\item\texttt{Adaptive methods: } Every document is different. Some techniques might work for a few and corrupt other. To determine which technique to use for each document image would be a useful feature in most cases. This is the exact problem that adaptive methods try to solve. Their approach is to analyze the document image surface and determine the most useful method. (hybrid switching? asi sa zamotam). More information on this subject can be found (ref adaptive) (for more information on?i)

\end{itemize}

Different kinds of improvements have been created in the form of new methods. They mostly consist of combinations of already mentioned methods, or have a similar basis. Further exploration of possibilities was, for example, done by (ref), (ref) or even in the articles already mentioned in this section.


- further possibilities - http://www.busim.ee.boun.edu.tr/~sankur/SankurFolder/Threshold_survey.pdf

- adaptive - http://www.mediateam.oulu.fi/publications/pdf/24.p

-otsu - https://www.researchgate.net/publication/277076039_Image_Binarization_using_Otsu_Thresholding_Algorithm

- global thresholding https://www.researchgate.net/publication/236029409_Combining_Multiple_Thresholding_Binarization_Values_to_Improve_OCR_Output

- thresholding - https://ieeexplore.ieee.org/stamp/stamp.jsp?arnumber=7784945

ref - why binarize https://www.researchgate.net/publication/287206352_Prediction_of_the_Optical_Character_Recognition_Accuracy_based_on_the_Combined_Assessment_of_Image_Binarization_Results

\subsection{Noise Reduction}

Image noise is an usually unwanted random variation of brightness or color information in images, reminding film grain in digital cameras. It is often caused by the sensor of a scanner or a digital camera attempting to record small amounts of light. 

For OCR engines, processing noise can be quite tricky. Its presence is especially crucial during binarization and edge detection. Edge detection is based on the differences between the "outside" of the element and its "inside". When noise is present, it can disrupt these edges and the exact (boundary?border?) is therefore changed and hard to detect.

There exist different type of noise:

\begin{itemize}
\item\texttt{salt and pepper noise - } also known as impulse noise, it can be recognized as randomly occurring black and white pixels. It is caused by sharp and sudden disturbances in the image signal, and can be most likely seen on old photographs.

\item\texttt{statistical noise - } is an unexplained variability within an image. It is characterized by a probability density function. Most common is \emph{Gaussian noise}, \emph{Shot noise}, \emph{Quantization noise} or film grain.

\item\texttt{other} - like anisotropic or periodic noise. These are often found on old photographs or documents.

\end{itemize}

Given the type of noise, different approaches are used for its reduction. To name a few of the most popular (zdroje? ):
\begin{itemize}

\item\texttt{linear filters - } like \emph{mean filter}, \emph{Gaussian filter} or \emph{Wiener filter}. As described in (zdroje?), these filters are usually simple and easy to implement, fast and therefore widely used. They work on averaging the color values of neighbour pixels, so they can help assure an averaged view of the contrast along regions of different color. This, however, rather than correcting the color of the affected pixel, results in creating an area with an average (and incorrect) color. This blurs the edges and smooths out other fine details.

\item\texttt{non-linear filters - } (zdroje o median filtry) the most popular is \emph{median filter}. It works similarly like a mean filter, but for each pixel, instead of replacing its value with a mean of surrounding pixels, it calculates the median. In comparison with linear filters, it provides considerably sharper images and in cases preserves the edges. The downside is the increased run time, and with high level of noise, results are almost the same as with linear filters. However, it is highly(??) effective on salt and pepper noise.

\item\texttt{non-local means method} - (mentioned in ..zdroj... ocr article) takes a mean value of all pixels, weighted by how similar these pixels are to the target pixel. This method produces an even sharper and clearer image results as mentioned filters, .... zase spomenut cas

\end{itemize}

What combination of techniques to use for the optimal results greatly depends on the input image - if there is little to no noise, de-noising an image can cause more harm than good. Therefore, it is always important to approximately compute the signal:noise ratio before applying any kind of filters.

To calculate this properly is a more complex task than thought. Although simple algorithms have already existed for a few years (examples, zdroje), they return poor results and most of them are almost not worth using because of the time complexity. They are mostly based on comparison - testing the difference between each pixel and its neighbors for various threshold numbers based on the dynamic range and calculating deviation, picking "windows" of different sizes and calculating their means and deviation, based on which we decide whether the center pixel is corrupted or not etc...

For a more complex (and sufficient) solution, we have to run many more mathematical calculations on the input image. These calculations require a lot of knowledge in the field of probability and statistics, specifically probability distribution. The basic idea for these noise recognition algorithms is \emph{separating frequencies}. This is usually done by dividing the image into blocks and running a transformation (for example a \emph{discrete cosine transformation}) on each of the blocks. The frequency components are then grouped by increasing frequency. Then, the extraction of statistical features (such as kurtosis, mean, standard deviation) is performed. (ako presne funguju -> zdroje)

Noise is most likely to be found on higher frequencies and we use this observation along with our computed statistical features to determine the thresholding on higher frequencies. If the threshold of the input image is greater, the image is determined to be noise-free. 

\subsection{Scanning border Reduction}

Scanned documents often have visible dark borders (or \emph{marginal noise}). These are created either during scanning (due to the presence of neighbouring pages), or are an unwanted result of the binarization process.

Similarly to noise reduction, running this step could only be harmful if there is no marginal noise present. Therefore, the algorithm has to have two steps - \emph{marginal noise detection} and, if successful, \emph{marginal noise reduction}.

Marginal noise is present as a dark connected component of the page. This is a fact that most detection algorithms take advantage of. Firstly, they extract the connected components of the page. This can be either done  by a \emph{black filter} (a moving rectangular window that calculates the ratio of black pixels on border zdroj), a sequential algorithm, a two-pass algorithm... After obtaining the connected components, different heuristics are used - most based on removing the connected components near the edges. In the end, usually a "cleanup" is performed (like \emph{white filter} - rovnaky zdroj ako black) - for removing all unnecessary elements around the borders (an unwanted text from the neighbour page, noisy components in the corners...).
The results of this algorithm depend greatly on the values of thresholds, margings and other hardcoded parameters. Different images work well under different values. This is why the cleanup part's parameters are usually dependant of the results of previous parts.

Another approach is the one mentioned in (zdroj edge detector). It is based on edge density calculations, and the fact that text areas have generally a much lower density than edge areas. An edge detector (in this case, sobel - zdroj) is used for examining vertical marginal noise, and later horizontal marginal noise. It returns the positions of the margins. Anything beyond these positions is detected as an unwanted noise, and can be therefore removed by filters or other removal algorithms.

edge detector - hhttp://citeseerx.ist.psu.edu/viewdoc/download?doi=10.1.1.60.3969&rep=rep1&type=pdf

black filter -  https://www.researchgate.net/publication/224100168_A_simple_and_effective_approach_for_border_noise_removal_from_document_images
sequential - https://www.rug.nl/research/portal/files/14632744/thesis.pdf

\section{Basic Techniques}

Nowadays, two approaches are proposed for OCR. The first one is an approach based on heuristics - page segmentation, line and element detection and only after that, character detection, which is based on different approximation and sometimes naive algorithms. The second one is an approach that is widely used and researched today - and that is the use of neural networks.

In this paper, we will mostly be concentrating on the heuristics approach of the OCR problem. However, deep learning also deserves a mention, and we will also briefly discuss this method in this section.

A lot of OCR engines nowadays differ in the way their heuristic approaches work. They all agree on one thing, though, and that is the succession of steps that need to be implemented. Firstly, a page segmentation has to be performed. This segments(divides?) the pages for determining the elements of the image. Now, the recognition algorithm could simply take place. However, most of the existing OCR engines (like Tesseract or ....) "pre-process" the output of page segmentation. They use steps like \emph{representation} and \emph{feature extraction}, which are based on simplifying the extracted elements of image documents. This contributes to better results of the OCR, and, last but not least, helps to improve the speed of the algorithm.

Only after this, the character recognition algorithm is applied.

We will go over these steps in the following individual subsections.

\subsection{Page Segmentation}

In many cases, OCR system heavily depends on the accuracy of page segmentation process [ref1]. There are three main approaches[ref2] to this problem - either we start with an image as a whole and recursively divide it into parts until criterion is met, or we start with individual pixels and group them according to similarity of their attributes, or we use a combination of these two methods.

In this section, we are going to introduce a few concrete algorithms that are based on these methods and improve them with their own heuristics:

\begin{itemize}
\item\texttt{X-Y cut } algorithm, also referred to as \emph{RXYC algorithm}. It is a tree-based algorithm, where the node of the tree presents the whole image page and leafs of the tree the individual segments. Algorithm starts by splitting the root of the tree into rectangular parts. Every step of the recursive algorithm includes projection profile and threshold computations, depending on which the node will either be (decided?) to be the end segment, or a split into two other rectangular parts will be executed. The algorithm finishes once there are no more nodes to split.

Although this algorithm is easy to implement and pretty fact, according to the benchmarking test of [4], it should not be used on other images than clean, properly skewed, with no noise whatsoever. In the presence of any imperfections, it usually takes the whole image as a segment.

\item\texttt{Smearing } algorithm, also referred to as run-length smearing algorithm (RLSA). It works with binarized image (which should not be a problem if preprocessing of image is present), trying to link together black areas that are no more that C pixels away from each other. This algorithm is applied both row-wise and column-wise, and the resulting bitmaps and then combined.

The downside to(of?) this algorithm is that it classifies text-lines merged with noise blocks as non-text. Although its results turn out better than those of RXYC algorithm, it is still not usable for classical text-documents. However, it is widely popular among vehicle. plate recognition.

\item\texttt{Voronoi-diagram based algorithm} is an algorithm based on connected components. It creates a Voronoi diagram from sample points extracted from the edges of individual connected components. It then "cleans up" the created diagram - deletes the edges that pass through a connected component, or the edges with its associating connected components that have a small distance or a too large area ratio.

This method provides better results for image documents with noise. However, it does not work well on images with different fonts or styles, as the spacing is estimated from the document image. This results in over-segmentation errors, and it is advised to use this method on a homogenrous collection of documents.

\item\texttt{Whitespace Analysis: } This algorithm is based on the assumption of white background (which, ones again, should not be a problem). It firstly tries to cover the the background with the union of white rectangles. These covers are then sorted in respect to a sort key based on a weighting function, which has the purpose of assigning higher weight to taller and longer blocks. Then, covers are added to form background cover until the sort key is less than the threshold.

Most of the time, whitespace analysis provides satisfactory results. However, it is very sensitive to the stopping threshold, therefore results in over-segmentation or under-segmentation.

\item\texttt{Constrained text-line detection} approach is similar to whitespace analysis, however, it starts with finding the whitespace rectangles that cover the page background in order of decreasing area. From these rectangles, columns separators are chosen based on their properties - aspect ratio, width, and proximity to text-sized connected components.

The upside of this approach is that it is nearly parameter free - therefore, it is a good option for documents with different font sizes or layouts.

\item\texttt{Docstrum} algorithm is also based on extracting connected components. It then performs a near-neighbor clustering. We already mentioned a similar approach in [deskew] section, and this algorithm also depends on the skew of the image. Firstly, connected components are divided into two groups - one with characters of the dominant font size, the other with characters from titles or section headings. For each connected component, its nearest neighbours are found. The algorithm then determines a histogram of distances and angles between nearest neighbors. Peak of the angle represents the dominant skew, which is used for determination of a within-line nearest neighbor pairs. These pairs are then merges into text-lines, which are merged into blocks.

The downside to this approach is similar to the one in Voronoi diagram, and that is spacing variation. Due to different font sizes and styles, this can lead to over-segmentation. Therefore, it is advised to use this algorithm on a homogeneous collection of documents.

\end{itemize}

Other page segmentation algorithms approach this problem with a few other steps. For example, Tesseract OCR engine first finds different lines of the image, and then fits text to them. In section (...), we will later discuss and take apart this method (or here?).


In the Voronoi and docstrum algorithms, the inter-character and inter-line spacings are
estimated from the document image. Hence spacing variations due to different font sizes and
styles within one page result in over-segmentation errors in both algorithms. For instance,
in many cases, they fail to estimate the inter-line distance correctly, and hence split the
zones into individual text-lines, resulting in a large number of over-segmentation errors.
The number of segmented zones for these two algorithms is much higher than the number
of zones in the ground truth. In some cases, text-lines in page title are incorrectly segmented
(see Figure 7) due to large variation in font size.

ref1: http://www.music.mcgill.ca/~ich/classes/mumt611_07/Evaluation/MaoKanungo2001.pdf
ref2: http://ijcsit.com/docs/Volume%204/vol4Issue3/ijcsit2013040307.pdf
ref4: https://www.researchgate.net/profile/Thomas_Breuel/publication/5431819_Performance_Evaluation_and_Benchmarking_of_Six-Page_Segmentation_Algorithms/links/551dd6dd0cf213ef063eb1ee.pdf
xy cuts: G. Nagy, S. Seth, and M. Viswanathan, “A prototype document
image analysis system for technical journals,” Computer, vol. 7,
no. 25, pp. 10–22, 1992.
smearing: K. Y. Wong, R. G. Casey, and F. M. Wahl, “Document analysis
system,” IBM Journal of Research and Development, vol. 26,
no. 6, pp. 647–656, 1982.
Whitespace: H. S. Baird, “Background structure in document images,” in
Document Image Analysis, H. Bunke, P. Wang, and H. S. Baird,
Eds., World Scientific, Singapore, 1994, pp. 17–34.
Docstrum: L. O’Gorman, “The document spectrum for page layout analysis,”
IEEE Trans. on Pattern Analysis and Machine Intelligence, vol. 15,
no. 11, pp. 1162–1173, Nov. 1993.
Voronoi-diagram: K. Kise, A. Sato, and M. Iwata, “Segmentation of page images
using the area Voronoi diagram,” Computer Vision and Image
Understanding, vol. 70, no. 3, pp. 370–382, June 1998.
Constrained text-line finding:T. M. Breuel, “Two geometric algorithms for layout analysis,” in
Document Analysis Systems, Princeton, NY, Aug. 2002, pp.
188–199.
Whitespace Cuts by Shafait et al
tesseractRef: https://static.googleusercontent.com/media/research.google.com/en//pubs/archive/33418.pdf

\subsection{Representation and Feature Extraction}

Upon extracting elements by page segmentation, we are left with an unnecessary amount of information about each element. We could feed all these information into a recognizer. This would, however, decrease accuracy of the recognizer and significantly increase its time complexity. For these reasons, a set of features is extracted for each element (class?) that distinguishes it from other classes while keeping characteristic differences.

There are various approaches to representation and further feature extraction. As described in [super kniha], these representations can be either:
\begin{itemize}

\item\texttt{Global transformation and series expansion:} These methods are based on the representation of continuous signals by a linear combination of simpler functions. The coefficients of the linear combination provide a compact encoding known as transformation or series expansion. Deformations like translation and rotations are invariant under global transformation and series expansion. (nakopirovane, pochopit, preformulovat)
This approach includes methods like \emph{Fourier} or \emph{Gabor} transform, \emph{Fourier Descriptor},\emph{Wavelets}, \emph{Moments} and \emph{Karhunen-Loeve expansion}.

\item\texttt{Statistical representation} of an image uses a statistical distribution of points to take care of font and style variations to some extend. It involves methods like \emph{zoning}, \emph{crossings and distances} and \emph{projections}.

\item\texttt{Geometrical and topological representation }, where the various properties of characters are represented by geometrical and topological features. These features are based on the basic line types that form the character skeleton. An output of feature extraction process is a feature vector. This representation and feature extraction technique includes \emph{extracting and counting topological structures}, \emph{measuring and approximating the geometrical properties}, \emph{coding} and \emph{graphs and trees}.


\end{itemize}

(napisat k nim viac?)

To sum up, the goal of any representation and feature extraction technique is to create a "skeleton" for each character by selecting the most representative information from the raw data which maximizes the recognition rate with the least amount of elements. However, these are still a number of differences between the features. The most noticeable difference is reconstructability - for some methods, the image can be reconstructed to its previous version solely because of features. Others, like statistical representation, lose information about the original image and would need complicated approximation algorithms for (backwards ...?). Another difference is the transformations to which features are invariant.

More about different representation and feature extraction can be found [refs] or [].

super kniha ref: 
https://www.researchgate.net/profile/Arindam_Chaudhuri2/publication/321518201_Optical_Character_Recognition_Systems_for_Different_Languages_with_Soft_Computing/links/5a5122e20f7e9bbc10543023/Optical-Character-Recognition-Systems-for-Different-Languages-with-Soft-Computing.pdf
druha ref: http://www.ee.bgu.ac.il/~dinstein/stip2002/FeatureExtractionReviewTrierJainTaxt95.pdf

\subsection{Character Classification}

OCR engines widely use the methodologies of pattern recognition, which assigns an unknown sample into one of its predefined classes. There exist various approaches dealing with these methods, which are not necessarily independent of each other. Most often than not, OCR engines combine multiple methods to achieve the most accurate results.

The first and most simple way to approach classification is by \texttt{Template Matching}. It is based on the existence of predefined \emph{templates} - multiple bitmaps containing characters of the alphabet. Improved version of this method have an extended database of templates, including numbers and special characters. Once a character is detected, it is passed to the algorithm and for every existing template, it is similarity ratio is calculated. The template with the greatest ratio is then assumed to be the recognized character. This method has various implementation depending on how the ratio is calculated - for example, cross correlation, normalized correlation or euclidean formula can be used.

Although the implementation of this method is very simple, even small disfigurements and noise can greatly affect its efficiency. Also, in this case, a feature extraction would be unnecessary, as all templates are created manually.

To make us of the previous steps, \texttt{statistical techniques} come in handy. They are based on statistical modeling of the data. To determine the output, extraction of the features from input image (such as size, shape, intensity) is first performed. Then, with the help of statistical decision functions, these features are compared to the statistical model.

The problem with these algorithms is that they have no information about whole-part relations. For this reason, a newer approach has been tested out over the past few years - \emph{machine learning}.

\texttt{Machine Learning} is a method of data analysis based on artificial intelligence. Over time, it builds models of "training data". Based on them, it makes decisions and predictions on its input. 

In the case of character classification, training data is created by passing various different characters into the engine and also providing it with the correct output. In this way, the engine learns how different characters should look and applies this knowledge to its input.

This method is for example used in a one of the simplest machine learning algorithms, and that is the \texttt{KNN Classification Algorithm}, or K-Nearest Neighbor algorithm. The input image is compared to the training data and chooses its K nearest (objects) (objects that are most similar). After that, the image is classified with being assigned to the class most common among its K nearest neighbors.

More complex methods include the \texttt{Support Vector Machine algorithm} (SVM). This approach divides the data received into two classes - training and testing data. The goal of SVM technique is to deliver a model that predicts the output of the test set.

The learning is done by a \emph{SVM kernel}. The task of the kernel is to take the input data and transform it into the desired form using a combination of mathematical functions. Other than kernel functions, there are other parameters that tune the output of the SVM algorithm. \emph{Margin} parameter tells the engine the separation of line to the closest class points. \emph{Gamma} parameter decides how far the influence of a single training example reaches and \emph{regularization} parameter controls the misclassification of elements.

Many experiments have been executed in the field of machine learning for character classification. However, none of them can guarantee the accuracy of different approaches, as they all greatly depend on the training data. The more training data the machine can get, the more accurate it is. 

?? (end somehow, maybe talk about deep learning more and delete its subsection)

template: https://pdfs.semanticscholar.org/b5ac/80e654108b898a9fcc827eadf1580d500bcc.pdf
ref1: https://arxiv.org/ftp/arxiv/papers/1101/1101.2491.pdf
ref2: http://www.ijera.com/papers/Vol%201%20issue%204/BQ01417361739.pdf
ref3: http://ijarcet.org/wp-content/uploads/IJARCET-VOL-1-ISSUE-4-131-133.pdf

\subsection{Deep Learning}

\section{Available Implementations}

Over the past few years, OCR has become a part of our everyday lives. The demand for a reliable OCR engines has therefore risen, which lead to many new implementations or improvements of the already existing ones.

Most people take the expression "optical character recognition" to mean solely text recognition. This term, however, has a wider meaning - it also includes the recognition of other document elements, like images, forms, tables and many more.

There exist few OCR engines that provide all the features an OCR software should have - such as different types of preprocessing, support for all file extensions, recognition of different fonts, including handwritten documents - this is not needed in most cases. For example, customers using OCR for ticket validation do not need to process handwriting or different fonts, and customers trying to achieve automatic number plate recognition do not need to process tables or forms or any other elements.

For this purpose, a lot of OCR engines focusing entirely on one or more cases were implemented, as it is less complicated, time consuming and does the job. However, all of these existing OCR engines, however difficult, have one thing in common - they all have to recognize text elements to some extent.

It this chapter, we will go over the most popular and widely used OCR engines that focus (among other things) on word and character recognition. We will also discuss engines used for the preprocessing part of the algorithm.

\subsection{Tesseract}

Originally developed by Hewlett-Packard CO around 1990 [ref], Tesseract is one of the most robust and accurate OCR open-source engines. When it was firstly developed, Tesseract could only accept TIFF images containing simple one-column text in only English language. Since then, however, it has undergone a lot of improvements and added many features. As of today, Tesseract supports multi-columned documents (via its page-layout analysis), claims to support over 100 languages (including right-to-left text such as Arabic or Hebrew), works on different input and output image formats (with the help of Leptonica [ref] library) and is available for Windows, Linux and even Mac OS. In its latest version (4.0.0), Tesseract also added a new neural network (specifically a LSTM network) focused on line recognition. 

Tesseract does not have a GUI and works as a command line program. It is used mostly for development purposes and provides an OCR engine - libtesseract - and/that? gives developers a chance to create their own applications using tesseract API. Also, Tesseract contains no preprocessing algorithms. It advises users preprocess the input images themselves [ref].

This is the reason why many wrappers and other 3rd party projects using Tesseract have been created [reflist]. These projects mostly focus on creating GUIs for Tesseract or adding preprocessing algorithms to make the Tesseract engine more user-friendly. Although Tesseract is still in the process of development, it plans on doing no such thing and its future work consists mainly on focusing on LSTM networks.

Tesseract is one of the few OCR open-source engines. That is why it is so widely used for development and research purposes. However, for quick, everyday and user-friendly purposes, this is not the choice to go with.

reflist: https://github.com/tesseract-ocr/tesseract/wiki/User-Projects-%E2%80%93-3rdParty
offical overview: https://github.com/tesseract-ocr/docs/blob/master/tesseracticdar2007.pdf


\subsection{OpenCV}

OpenCV is an open source computer vision and machine learning software library. It claims [ref webpage?] to have more than 2500 optimized algorithms used for face detection and recognition, 3D object manipulation, photography editing (like red eye removal), tracking of moving objects or camera movements and other image processing functions.

On contrary to Tesseract, OpenCV is a user-friendly library. It is widely used as a framework for creating OCR applications and contains a lot of preprocessing functions. These can be then passed to the recognition algorithm, which makes the whole process of recognition easier and simpler. However, its main area of expertise is not OCR and definitely not document manipulation. For recognition, it mostly applies the \emph{template matching} technique. This technique is simple and easy to implement and works very well on ticket validation, credit card recognition or car plate recognition [ref]. Although these are all areas that OpenCV is widely used for with great success rate, the outputs from text document OCRs are poor [ref? to find]. It does not even have a library specialized on OCR.

The reason for mentioning OpenCV is exactly for its vast variety of preprocessing options. There are many projects[ref] that incorporate the work of OpenCV and Tesseract to achieve the most accurate and, most importantly, user-friendly results. Although OpenCV is mostly used via its Python interface, it also has a C++ and Java interfaces which can connect with Tesseract either directly via its C++ API, or through a wrapper like PyTesseract [ref].

\subsection{ImageMagick}

Even though it is not used for OCR and has no feature that would allow such a thing, ImageMagick is a software worth mentioning. In case of OCR, it is used widely for the preprocessing part of the work as well as OpenCV. It is a free, open-source software suite containing numerous features useful for text image processing, like transformations, color management, large image support, format conversion (it claims to be able to process over 200 different file formats), noise reduction,  and many more, mentioned in [imagemagick stranka?]. Its functionality is mainly utilized from command-line environment. ImageMagick also has various features that enabled users to create and modify images via a language interface like Magick++(C++) or PythonMagick(Python), but these interfaces are not as rich as the library itself. 

Although they might be used for the same ....(purpose?) in case of OCR, ImageMagick and OpenCV are ... different programs. OpenCV's goal is computer vision algorithms, like object identification and recognition, while ImageMagick is used solely for image manipulation - that is, image processing. This focus of ImageMagick leads to better accuracy of its results. The downside of this is that ImageMagick can only apply functions to images and does not make assumptions about images itself. This leads to a more complicated integration of the software with Tesseract or other OCR engine.

\subsection{Commercial Software}

On contrary to Tesseract, there exist many OCR softwares that are used for commercial purposes. Although these are practically useless for developers, they produce pretty satisfactory results, in many cases better than open-source OCR engines.

In this section, we will mention few of the most popular ones, like \emph{ABBYY FineReader},\emph{ReadIRIS} or \emph{Omnipage}. We will look at their comparison report and mention a few of their features.


We will start this chapter off by looking more closely into \texttt{ABBYY FineReader}. This OCR engines claims to convert scanned PDF files into editable electronic formats like MS Word, MS Excel, RTF, HTML etc... It supports over 192 languages and even has a built-in spell check for 48 of them and includes the support for table and spreadsheet recognition, as well as batch processing of multiple documents. It also supplies SDKs for embedded and mobile devices. These are all the reasons why, to this date, it has over 20 million users. However, as described in [comparison] report, this engine has also its downsides. In some cases, Tesseract seemed to perform significantly better than ABBYY FineReader. This was mostly in the cases of Gothic fonts and good quality images, which FineReader had trouble processing. Moreover, FineReader seemed to have a problem with pages with great amount of small characters, as well as pages with small about of big characters, although the engine was trained to recognize them. However, when dealing with either noise or complicated segmentation, Tesseract mostly failed, while FineReader performed quite nicely.

Another software we would like to discuss is \texttt{ReadIris}. It provides similar features to ABBYY FineReader - batch processing of documents, conversion to MS Word, MS Excel or MS Powerpoint, splitting or merging PDFs and Calc table recognition. It also includes features like voice annotations. Its main difference from ABBYY FineReader is that it is less robust - it is not available for Linux, has only 138 languages compared to FineReader's 192 and the number of output formats is undoubtedly smaller. Although this might be a downside for user experience, the performance of ReadIris can sometimes even beat the one of FineReader. For example, in a study done by ... on card recognition, ReadIris performed better than FineReader in both numeral and orientation detection. ReadIris is also widely used for the recognition of Arabic and Hebrew characters, where it performs even better than ABBYY FineReader. [find ref?]

Last, but not least, we are going to have a look at \texttt{Omnipage}. It promises its customer the same things that both already mentioned softwares - conversion of scanned documents into editable documents with various formats. In comparison to ReadIris, it supports similar number of language. However, its con is the availability for Linux and greater amount of supported formats. As mentioned in [omnipage test], Omnipage has had great trouble working with colored images. It firstly transforms them into greyscale and only then performs the recognition. It also has problems dealing with rotation over 12 degrees, which simple preprocessing algorithms in combination with Tesseract should not find to be a problem. On the other hand, Omnipage provides a user-friendly environment and is optimized for speed, which contributes to the overall user experience.

These are by far not the only existing engines. When choosing a commercial OCR software, numerous other options promising similar (features??...) and results arise - like \emph{SimpleOCR}, \emph{Orpalis}, \emph{Adobe Acrobat Pro DC} or \emph{CVision OCR Engine}. However, these programs have almost no value for developers (except for comparison studies), whose work is still either concentrated on improving and expanding the Tesseract engine, or based on Tesseract's robust library.


omnipage test: http://citeseerx.ist.psu.edu/viewdoc/download?doi=10.1.1.12.2361&rep=rep1&type=pdf
readiris: http://www.irislink.com
cardCompar: https://www.researchgate.net/profile/Ilia_Safonov/publication/265799485_Intellectual_Two-sided_Card_Copy/links/551d656a0cf2bb3a536b3d54/Intellectual-Two-sided-Card-Copy.pdf
comparison: Report on the comparison of Tesseract and ABBYY FineReader OCR engines, Marcin Heliński, Miłosz Kmieciak, Tomasz Parkoła, Poznań Supercomputing and Networking Center, Poland
charts: https://www.simpleocr.com/Compare-OCR-Software

// mention somewhere that .tif images are the best
\chapter{Layout Recognition For Tabular Data}

To extract elements of a table from an image document, the table first needs to be detected. This is no easy task.  Although for many people, a table is simply a collection of vertical and horizontal lines which can be detected quite easily, this is usually not the case. Often, either the borders are partially or completely missing, the table contains cells of different sizes, multiple columns or rows are merged together on only some places… These and many other factors often cause the table to be of a different form than the well-known matrix. 

Furthermore, complications arise when the image document does not correspond to the typical one-column, graphics-free layout. This often causes complications of reading order and therefore confusing results of text recognition, errors with table detection as spaces between columns can be interpreted as table column borders, and, if tables, forms or other graphic elements are misinterpreted as simple text lines, a text page without any contextual sense. Therefore, in most cases, a \emph{layout analysis} first needs to performed.

\section{Layout Analysis}

\textbf{Layout analysis} is the process of identifying and categorizing image document elements, such as figures, tables, forms, math symbols, headers, footers or simple paragraph (\emph{geometric layout analysis}) text and semantically labeling them according to their logical roles (\emph{logical layout analysis}).

In the previous chapter, we already covered the basics of \textbf{geometric layout analysis}, which is the same process as \emph{page segmentation}. The output of this process is a data structure (like vector) containing all of the detected elements. Depending on the configuration of the analysis, the resulting elements can be of various types, such as characters, words, text lines, or even tables and forms. Usually after this, logical layout analysis is applied.

\textbf{Logical layout analysis }is used to determine the reading order of the image document. Its design requires carefully choosing a document structure representation for capturing even the most complex documents. The result of logical layout analysis is often in a form of mapping of each element to its corresponding label. These labels provide an information about the semantical order of the document elements. For example, a label could represent simply the ordering, or it could contain more complex information, such as "table header", "page footer"\ldots

To determine the correct labels is sometimes hard even for a human eye. With various differently aligned columns with different font sizes, or with image captions appearing on different sides of the image in every document, people often determine which elements belong together only according to their intuition (e.g. when reading about a recent earthquake, caption saying “Rescued puppy” probably belongs to the picture of a dog instead of a flooded beach, although in can be placed right in the middle of these two images). Computer has no notion of such things. This is often a cause of many errors and a reason why a lot of OCR engines claim to work on only documents with specified layouts, e.g. single-column, non-graphical\ldots

Various heuristics are being used for determining labels\cite{logicalLayoutTemplate}. In this section, we will be concerned with the few of the most widely used.

\begin{description}
\item[Templates] The most simple and basic approach is the technique of the already mentioned templates. It is based on simply mapping the input document to already predefined template, which already has all the information about the structure of elements. Although a naive approach, in the OCR engines used widely for processing a single type of documents (such as ticket validation, recipe or passport recognition, recognition of forms filled out by patients in hospitals), there is no use for anything more complicated, as this process yields almost perfect results.

\item[Rule-Based Approaches] A human reader often determines the logical succession of document elements by font settings and locations of the element. Rule-based approaches take advantage of this fact and create heuristic \emph{rules} that determine the type of the element. For example, a rule for a page header could be "has the smallest y-axis value, has font size above 22pt, is bold, and is the only element on its line".

\item[Syntactic Methods] In this approach, the structure used for element labeling is in the form of a set of formal (usually context free) grammars. These grammars contain rules for aggregating pixels into more and more structured entities until they form logical objects. From these grammars, parsers for the syntactic analysis are automatically obtained. They are then used to perform the actual labeling of the detected elements.

\item[Machine Learning] Already mentioned in this thesis, machine learning methods are widely popular among document recognition. For logical ordering, these methods are even more powerful --- given enough information, the network determines the labeling on its own, without the need of complicated heuristics. However, neural networks need to be trained. This is where different types of machine learning techniques are distinguished. For example, the neural network can be given a set of rules and input images, which leads the learning process closer to the results similar to the human observation. Also, it can be solely reliant on raw physical data and itself.

\end{description}

Worth mentioning are also techniques like \emph{Blackboard system} or \emph{Description language} or methods based on \emph{Hidden Markov Models} \cite{logicalLayoutOther}.

Layout analysis is a crucial part of almost every OCR engine. If either geometric or logical analysis fails, the OCR engine will have corrupted input data, which will lead to significantly lower accuracy of the recognition process. 

In this thesis, we focus only on table recognition. Our concern is not the headings and fonts, headers and footers of the image, but only the tabular data. However, this does not mean that layout analysis is unnecessary. We still need to extract the tables from page layout and ideally include also their headers, footers or any other information the table may be containing.

\section{Table Detection}

The goal of table detection is to determine \emph{if} the table even is present on the page, and if yes, \emph{where}. The result should be a meaningful representation with the information about the table's location. After detection, the table is then passed to a recognition algorithm.

The problem of table detection is challenging due to variable layouts and random positioning of table elements. A basic table (like the default MS Word table) having a default grid layout, with multiple perpendicular horizontal and vertical lines, is not that hard to detect. Its borders could be detected by a simple line detection algorithm (for example Hough transform). However, there are many problems that can complicate this process, like invisible borders, invisible lines, different cell sizes, multiple merged cells..

\subsection{Existing approaches}

Over the years, there have been many heuristic approaches presented for table detection. In this chapter, we are going to have a look at a few of them and determine what they have in common, as well as point out their pros and cons.

Presented as one of the first table detection algorithms by \citet{TRecs}, the \textbf{T-Recs} table recognition system is based on a bottom-up approach of clustering word bounding boxes and building a "segmentation graph". This results in creating different regions of page, and these are then evaluated according to certain criterion. If they satisfy them, the region is determined to be a table. Although widely used in the future, this techniques has a few setbacks. T-Recs is controlled by a set of numerical parameters, which result in different results of table detection depending on the layout that is given. These parameters need to be set manually. Moreover, it yields unsatisfactory results on multi-column documents.

Another algorithm was described by \citet{MediumTable}. In single-column documents, a page can be easily segmented into individual text-lines. The table detection problem is then perceived as an optimization problem, where the start and end text-lines belonging to a table are identified by optimizing some quality function. However, this approach fails on multi-column documents, or on documents having more than one table.

Problems during analysis of multi-column documents were approached by \citet{tableDetHeterogeneous}. Here, page segmentation process is performed by Tesseract via \emph{tab-stop detection} [odkaz na section hore?], which claims to work well on multi-columned documents. Based on the results of Tesseract's analysis, this algorithm then aggressively searches for text column partitions that could possibly belong to a table region. Although this process returns the desired table partitions, it also produces a lot of false alarms, like section headings, page headers, footers, equations and so on. A smoothing filter is applied to remove these unwanted partitions.

Detection of table columns is performed from these partitions. Vertically aligned partitions are simply grouped into a single column with further removal of columns with only one partition. Table columns then need to be grouped together to form a table. In this algorithm, a merge is performed only if at least one horizontal ruling is present between two columns.

Results of this algorithm have shown 86\% precision. The biggest problems have shown to be full-page tables, often resulting in over or under-segmentation, partial detection or false positive detection. 

\citet{tableDetectCesarini} describes another approach based on the document being hierarchically represented by a structure similar to a MXY tree. The presence of a table is determined by searching the tree for parallel lines, which contain white spaces and other perpendicular lines between them. Located tables can be merged on the basis of proximity and similarity criteria. However, this approach fails if no lines in tables are presented --- which is usually the case of many tables.

Another method is presented by \textbf{pdf2table} project \cite{pdf2table}. This method is based on assigning each text object of the page its positional attributes. Depending on them, text objects are then merged into single-lines (lines with only one text object), multi-lines (lines with more than one text object) and multi-line blocks (multiple multi-lines merged together). The table detection algorithm is based on merging multi-line blocks that may belong to the same table, with the help of a heuristical threshold that determines the greatest number of single-line objects between two multi-line blocks possible.

This method also assumes the input to be a single-column documents. However, a user can provide it with an information about the number of columns, which yields much more accurate results.

Multiple other approaches exist, each one of them working on different types of tables and yielding slightly different results. Worth mentioning is, for example, \emph{Sparse Line Detection} \cite{sparseLineDetection}, which however already uses principles based on machine learning. Some of the other methods briefly mentioned by \citet{otherDetection1} or \citet{otherDetection2}.

\subsection{Vymysli meno}

When it comes to table detection, 


\section{Table Recognition}

- available implementations? 







\chapter{Table Recognition Implementation}

This chapter gives us an overview of procedures and tools used to create a table recognition software. 

The implementation is divided into two main parts --- \emph{preprocessing}, which consists of parsing arguments from command line and image manipulation, and \emph{tabular OCR}, which calls a character recognition software and given the output data, determines the presence of tables in the input images.

\section{Preprocessing}

The preprocessing part of the implementation is mostly only an interface representing everything that could still be implemented. It consists of two parts --- a parser and a preprocessor.

\emph{The parser} is used for processing the input command line arguments. It also contains a configuration class, which saves the values of preprocessing arguments and filenames that need to be processed. This is later used in the preprocesser. 

The main function of the parser is a function of the configuration class \emph{parse\_args()}. It takes all of the command line arguments and initializes the configuration class with the given values. If no value has been specified, the current preprocessing option will not be implemented. The only exception to this is the command line option \emph{-p} of \emph{--preprocess}, which initializes the preprocessing option values to those that had the best results overall.

The \emph{parse\_args()} function also parses the filenames and directory names of the files that will be processed. The parser therefore also contains functions for file manipulation, like creating directories, subdirectories, determination of files or directories, extracting the filename from path and other. 

After the parsing part is done, processing of the individual images begins. This is done in a for loop for each file returned from the parser. Images are therefore processed one by one.

Although we need only one copy of the image for passing to the preprocessor and then processing, we keep another copy during the whole run of the program. Therefore, once the recognition algorithm is over, the result can be displayed on the original image to prevent the user from seeing the output of the preprocessor, as the preprocessed image may be too noisy, hard to read and will most likely lose most of the color information.

First step of the processing is the call of the \emph{preprocessor}, concretely its \emph{preprocess\_file()} function that proceeds by calling the individual preprocessing functions depending on the configuration settings. In this implementation, only the basic preprocessing options are available, e.g. enhancing, deskewing, greyscale conversion and binarization. These are all implemented in the Leptonica library which also provides the calls of these functions. 

The initial idea for preprocessing was to provide more complex functions which are included in the OpenCV library. However, this was not the goal of the thesis. Therefore, the preprocessing part stayed very simple and mostly demonstrative and the user is still advised to preprocess the images manually.

However, preprocessing is a crucial part when it comes to OCR. We will discuss its importance and the options of improvement (like usage of OpenCV, determination of the need to preprocess etc.) in the next sections \xxx{pridat odkazy}. 

\section{Tabular OCR}

The actual table recognition is executed in a \emph{process} file along with a \emph{utils} file that contains a few helper functions. The main function called is a function of a page class, \emph{process\_image()}, that executes our table detection algorithm.

This is also the place where the already mentioned Tesseract engine is integrated and called. In our algorithm, as well as the Tesseract table detection algorithm, we use the already implemented Tesseract line and symbol detection. Therefore, our algorithm relies greatly on the functioning of the Tesseract engine. 


\subsubsection{Algorithm implementation}

Our algorithm is based on the already mentioned Tesseract symbol and textline recognition and moreover on whitespace detection. Upon detecting whitespaces between individual symbols, we try to heuristically estimate the whitespaces between words and, furthermore, columns, for each textline of the image. Once we have all the textlines separated into columns, we try to merge consecutive lines with similar columns into a table.  

In this section, we will analyze this algorithm step-by-step and overview the functions used.

\begin{description}
\item[Textline initialization] The whole process begins with initialization of individual textlines. This is the only place where we use the character recognition from the Tesseract software.

We initialize the Tesseract API without the use of neural networks and obtain both lines and symbols from the API. Then, we iterate over all the lines and symbols and try to assign symbols into lines to which we think they might belong to. The result of this function is therefore a list of all textlines that contain the information about their individual symbols, like positioning and their actual value in UTF8.

This function runs in $O(m*n^2)$. The initial idea for this implementation was to firstly sort both the symbols and lines by their y coordinates (which is simply $O(\log n)$ and  $O(\log m)$) and iterate the cycle in $O(m*n)$ by simply iterating symbols and jumping to another line once symbol does not fit in the given line. This would mean a significant improvement in the time complexity. However, a problem occurred when iterating symbols. By default, Tesseract recognizes anything it can find and assumes it to be a symbol. This creates a lot of false positives, including noise recognized as dots, white spaces, and, most importantly, horizontal or vertical lines, like footer or header separators, underlinings of words, table borders etc. We tried to adjust our algorithm to ignore empty characters. Horizontal and vertical line detection, however, was a harder task. Although most of the lines are pretty simple to detect (either their width or height is significantly greater than the other, or either width or height is unusually small in comparison to other characters), there is no one criterion that would suffice all lines, with problems occurring mostly at thick but short lines. Therefore, we sacrificed \xxx{dobre slovo?} the time complexity in favor of accuracy.

This function is the place where most of the mistakes are made and time complexity is consumed. Although a robust software, Tesseract recognition is still far from ideal and sometimes fails at even the simplest images. Also, its recognition takes up significantly more time than all of the other functions combined. We will discuss the details of Tesseract recognition complexity and errors and their possible improvements in following chapters.

\item[Deletion of unnecessary lines]

As already mentioned, Tesseract recognition algorithm includes a lot of false positives. In this function, we delete all unnecessary lines, specifically:

\begin{itemize}
\item\textbf {Empty lines: } Like horizontal or vertical line segments, borders or other lines that contain no UTF8 symbols are of no use and are therefore deleted from the textline list.

\item\textbf {Table textlines: } Table textlines are parts of the image that already had a border around them, which might have been either a table, form or even a graphics image. Tesseract often recognizes these parts as single textlines. These textlines therefore contain multiple other textlines, and are significantly greater in height. We delete these textlines by simply looking at their height and the font of their symbols.
\end{itemize}

\begin{figure}[H]
\begin{subfigure}{0.45\textwidth}
\includegraphics[width=\linewidth,height=40mm]{img/implementation/textlineEmpty.png}
\caption{Empty lines} \label{fig:1b}
\end{subfigure}
\qquad
\begin{subfigure}{0.45\textwidth}
\includegraphics[width=\linewidth,height=40mm]{img/implementation/textlineTable.png}
\caption{Table textline} \label{fig:1c}
\end{subfigure}
\caption{Deletion of unnecessary lines} \label{fig:1}
\end{figure}

Once this step of the algorithm is done, we are left only with lines that contain symbols and can be a part of the table.

\item[Column detection]
Upon obtaining the textlines, we try to determine the all their their columns from the symbols they contain. This is done for each line individually. Firstly, we merge symbols into words. After that, we merge words into columns. Although we could simply just merge symbols into columns and ignore the whole processing of the words, this would leave us with no spaces between words in individual columns.

We start this process by getting all the spaces between individual symbols and sorting them by size. For a human eye, upon seeing this list, to determine the whitespace between individual words and columns is mostly a pretty easy task. Following are three lists of spaces and a visualized merge of our algorithm according to them:

\texttt{}{
1 1 2 2 2 3 3 3 3 3 3 3 3 3 3 3 3 4 4 4 4 5 15 15 16 16 18 18 227 235;

2 2 2 2 2 2 2 2 2 3 3 3 3 3 3 3 3 3 3 3 3 4 4 4 4 4 4 4 5 5 6 15 15 16 17 165 235;

1 1 1 2 2 2 3 3 3 3 3 3 3 3 3 3 3 4 4 4 4 4 4 4 4 5 5 15 15 15 17 17 17 18 134 235;}

\begin{figure}[H]
\hspace*{\fill} % separation between the subfigures
\begin{subfigure}{0.80\textwidth}
\includegraphics[width=\linewidth]{img/implementation/mergedOrig.jpg}
\caption{The original image} \label{fig:1b}
\end{subfigure}
\begin{subfigure}{0.80\textwidth}
\includegraphics[width=\linewidth]{img/implementation/mergedWords.png}
\caption{Merged into words} \label{fig:1c}
\end{subfigure}
\hspace*{\fill} % separation between the subfigures
\begin{subfigure}{0.80\textwidth}
\includegraphics[width=\linewidth]{img/implementation/mergedCols.png}
\caption{Merged into columns} \label{fig:1c}
\end{subfigure}
\caption{The process of merging symbols of a textline} \label{fig:1}
\end{figure}

\xxx{opat netusim ako formatovat, ma ten obrazok byt vobec tu?}

It is quite obvious that the word whitespace will therefore be around 5-6 in all cases and column whitespace 227, 165 and 134 respectively. In our program, we determined these whitespaces by iterating over all the spaces. Once we find two subsequent spaces that have "great difference between their values", we assume the greater one to be the whitespace of either words, columns or both.

So how do we determine whether the difference is too great? For both word and column whitespaces, we calculate a so-called \emph{multiplicator factor} and use it according to\xxx{like?} the next code snippet that is used for the determination of the word whitespace.

\begin{code}
for (it; it != all_spaces.end() - 1; it++)
{
	// get multiplication factor of current space
	double multi_factor = get_multi_factor_words(*it, constant);
	if (*std::next(it) >= multi_factor * *it)
	{
		// found the word whitespace at *std::next(it)
		// code to execute once the whitespace is found
	}
}
\end{code}

The simplest observation is --- the greater the current space is, the less the multiplication factor should be. Based on this, many different values and curves have been tried for the determination of multiplication factor. First observations from these attempts led to the estimation that the best curve to use would be logarithmic. However, the current implementation seemed to worked well enough and was therefore left as it is.

The determination of the column whitespace was done similarly. Although the function determining the multiplication factor was altered, the idea stayed the same.

Upon determining the column and word whitespaces, the only thing left is to merge symbols by these whitespaces.

The determination of whitespaces was probably the hardest part of the algorithm. There have been multiple different ideas for the implementation. The one that has been preferred most of the time was the idea of separating textlines according to their \emph{fonts} \xxx{tj. (aka, ako to napisem?) their heights}. The ones with similar fonts were assigned to same \emph{font category}, and the whitespace was then determined from the whole category. The word whitespace recognition worked slightly better with this approach. However, the determination of columns had a higher chance of failure, as the sizes of column spaces differed greatly and the algorithm had a problem with finding the point where word spaces end and column spaces start. Therefore, this simpler approach was chosen.

Another approach was to simply determine the size of the column space by a constant, e.g. word\_whitespace*constant = column\_whitespace. Suprisingly, the results of this approach were comparable to those of the current implementation. However, it was deemed to fail when it came to small fonts or full-page tables, and had no room for improvement in contrast with the current approach.

\item[Table creation]

Once we have the information about columns for each textline, we can start searching for tables. The table detection algorithm is done by a simple $O(n)$ algorithm --- iterating the textlines from top to bottom and merging two consecutive textlines together if they belong to the same table.

Following is an algorithm used to determine whether two lines represented by columns are in the same table:

\xxx{To Do - sformatovat? :( }

\begin{algorithm}[H]
\caption{Are textlines in same table}
\begin{algorithmic}[1]
\State $iter\_first$ represents the current place we are when iterating over columns of first line
\State $iter\_second$ represents the current place we are when iterating over columns of second line
\While{true}
\If {either $iter\_first$ or $iter\_second$ is at the end of their line}
\If {at least one pair of columns was found that should be merged}
\State \emph{merge()}
\EndIf
\EndIf
\If {current\_columns\_overlap() } \Comment{a function that checks whether the bounding boxes of the two columns overlap in the x axis}
\If {found columns do not overlap with other existing columns in the x axis}
\State save the position of overlapping columns for future mergeing
\EndIf
\Else 
\State increase either $iter\_first$ or $iter\_second$ depending on which is poiting to the box that has a lower x-axis
\State continue
\EndIf
\State $iter\_first \gets iter\_first+1$
\State $iter\_second \gets iter\_second+1$
\EndWhile
\end{algorithmic}
\end{algorithm}

Merge of the textlines is done by merging columns that have been detected to overlap, and adding other columns with no such attribute. In a typical n*m table, every column should overlap with the one underneath it, which is also mostly the case when running this algorithm.

Once we have at least two merged textlines, we use this new merged line as a current textline. Therefore, when creating a table, we append new lines to the already merged ones. At the end of this algorithm, our current table is therefore represented as a textline with the information about its columns (and a list of textlines that are in the current table).

\item[Output creation]

What we care about when creating an output are table cells. We create these by simply overlaying \xxx{dobre slovo? prelozit} rows and columns and saving their common areas as cells. The problem arises with the existence of multi-line rows, that is, rows that often span over multiple Tesseract recognized textlines. In our implementation, a simple constant-based algorithm is added to recognize at least some of them and therefore merge multiple textlines into one row. The algorithm for this could be improved differently, which will be discussed in the next chapter. 

Once we obtain cells, the only thing left is to create a user-friendly representation of the recognized data. Here, the user has two options according to the parameter he sets in the command line environment. 

The first option is a simple image output. Recognized cells are therefore bordered \xxx{dobre slovo?} by colored boxes in the original input image and saved in a PNG file.

The other option is a json structure of the recognized cells, which also contains text within each cell. The json has the following structure:

\lstset{
    string=[s]{"}{"},
    stringstyle=\color{blue},
    comment=[l]{:},
    commentstyle=\color{black},
}
\begin{lstlisting}
{"all_tables": {
  "cols": number of columns,
  "rows": number of rows,
  "table_repres": {
    "h": height of table,
    "w": width of table,
    "x": the x-coordinate where the table starts
    "y": the y-coordinate where the table starts
  }
  "cells": [
        {
        "box": {
           "h": height of current cell,
           "w": width of current cell,
           "x": the x-coordinate where the cell starts,
           "y": they y-coordinate where the cells starts,
        },
        "cols_no": in which column of the table the cell is,
        "rows_no": in which row of the table the cell is,
        "text": the UTF8 text displayed in the cell
        },
        ...
        other cells
    ]
  }
  ...
  other tables
}
\end{lstlisting}


By default, both output options are selected and therefore two files are saved in the newly created \emph{results} directory inside the build directory.

\end{description}

\begin{figure}[H]
\begin{subfigure}{0.45\textwidth}
\includegraphics[width=\linewidth]{img/implementation/implem1.png}
\caption{Individual textlines (blue) and their symbols (red)} \label{fig:1b}
\end{subfigure}
\hspace*{\fill} % separation between the subfigures
\begin{subfigure}{0.45\textwidth}
\includegraphics[width=\linewidth]{img/implementation/implem2.png}
\caption{Unnecessary lines} \label{fig:1b}
\end{subfigure}
\begin{subfigure}{0.45\textwidth}
\includegraphics[width=\linewidth]{img/implementation/implem3.png}
\caption{Textlines merged into columns} \label{fig:1c}
\end{subfigure}
\hspace*{\fill} % separation between the subfigures
\begin{subfigure}{0.45\textwidth}
\includegraphics[width=\linewidth]{img/implementation/implem4.png}
\caption{The resulting cells} \label{fig:1c}
\end{subfigure}
\caption{The process of table recognition} \label{fig:1}
\end{figure}

\begin{figure}
    \noindent
	\makebox[\textwidth]{\includegraphics[width=\paperwidth-100pt]{../img/implementation/programFlow.pdf}}
	\caption{Program Flow Diagram}
	\label{fig:mff}
\end{figure}

\chapter{Results}

In this chapter, we analyze the performance of our software and compare its results to those of the Tesseract \emph{table find} algorithm.

\section{}


\section{Comparison to Tesseract's tablefind}


\chapter*{Conclusion}
\addcontentsline{toc}{chapter}{Conclusion}

In this thesis, we explored and reviewed the existing approaches for optical character recognition, and focused on applying its results to the problem of table recognition.

As a main result of the thesis, we have developed a software package that combines the OCR functionality available in the Tesseract library with several methods of image preprocessing, and a newly developed heuristic-based algorithm that aims to improve the available possibilities of table extraction from scanned documents.

We have compared the results obtained from running this combination on a simple testing data set to the results from the TableFind algorithm of Tesseract. While both implementations required similar computational resources for processing the data, our implementation produced more accurate results for more complicated types of table layouts, especially on tables without available border separators, and in cases where the table layout depends e.g.~on subtle differences in cell formatting. Additionally, we have assessed how the resolution and information content of the input image affects the time complexity and output quality of the algorithms.

Despite the improvements in table recognition capabilities, we have observed that the outcome of the recognition still mainly depends on the quality of the input image, and is mostly improved by correct preprocessing.

In the future, we plan to improve the main deficiencies of the whole pipeline, primarily the mentioned preprocessing and several open problems with table recognition summarized in~\cref{sec:future}. From the observed results, we believe that after applying the preprocessing improvements, the table recognition system will produce results of excellent quality.

%%% Bibliography
\include{bibliography}

%%% Figures used in the thesis (consider if this is needed)
%\listoffigures

%%% Tables used in the thesis (consider if this is needed)
%%% In mathematical theses, it could be better to move the list of tables to the beginning of the thesis.
%\listoftables

%%% Abbreviations used in the thesis, if any, including their explanation
%%% In mathematical theses, it could be better to move the list of abbreviations to the beginning of the thesis.
%\chapwithtoc{List of Abbreviations}  -- this belongs to 20th century, abbreviations should be defined on the first use

%%% Attachments to the bachelor thesis, if any. Each attachment must be
%%% referred to at least once from the text of the thesis. Attachments
%%% are numbered.
%%%
%%% The printed version should preferably contain attachments, which can be
%%% read (additional tables and charts, supplementary text, examples of
%%% program output, etc.). The electronic version is more suited for attachments
%%% which will likely be used in an electronic form rather than read (program
%%% source code, data files, interactive charts, etc.). Electronic attachments
%%% should be uploaded to SIS and optionally also included in the thesis on a~CD/DVD.
%%% Allowed file formats are specified in provision of the rector no. 72/2017.
\appendix
\chapter{Software user guide}
\xxx{Jak to skompilovat v labu na unixu}

\xxx{Jak to spustit a co to ma za commandline args}

\xxx{Co je v kterejch zdrojovejch souborech}

\openright
\end{document}
