\chapter{Software user guide}

In this appendix, we provide a user guide for compiling and running the project.

The software has been tested on Windows~10 (MSCV 2017) and Ubuntu~18 (G++ 7.4). As we do not provide any binaries, the user must compile the project first.

To compile the project, execute the following steps:\\
\\
\textbf{Windows}:
\begin{enumerate}
    \item Clone GitHub directory from\\
    \url{https://github.com/todoval/Bachelor-s-Thesis.git}.
    \item Download and setup cppan.
    \item Run cppan in any directory.
    \item Build the project using CMake (make sure you have installed MSVC~2017 and CMake~3.8).
    \item Set the environment variable \texttt{TESSDATA\_PREFIX} to \texttt{tabularOCR/tessdata directory}.
\end{enumerate}

\textbf{Linux}:
\begin{enumerate}
    \item Clone GitHub directory from\\
    \url{https://github.com/todoval/Bachelor-s-Thesis.git}.
    \item Install the following libraries: libleptonica-dev, libtesseract-dev, libopencv-dev.
    \item Build the project using CMake (make sure you have installed G++~7.4 and CMake~3.8).
    \item Set the environment variable \texttt{TESSDATA\_PREFIX} to \texttt{tabularOCR/tessdata directory}.
\end{enumerate}

Follow these steps to run a sample demo:
\begin{enumerate}
    \item Take one of the pictures from our sample images directory at \url{https://drive.google.com/open?id=1cPPQc0H2AYUB7jHM8_6KwlQ05YNOrJdY}, for example \texttt{11-1.jpg}, and copy it to \texttt{your-build-directory/bin/Debug}.
    \item From this directory, run command:
        \begin{itemize}
            \item Windows: \texttt{tabularOCR.exe 11-1.jpg}
            \item Linux: \texttt{./tabularOCR 11-1.jpg}
        \end{itemize}
        In the rest of this user guide, we will be using a unified command call \texttt{tabularOCR}.
    \item The results can be found in a \\
    \texttt{your-build-directory/bin/Debug/results} directory as both \texttt{11-1.png} and \texttt{11-1.json}.
\end{enumerate}

\emph{Note}: The input image does not necessarily have to be in the same directory as the tabularOCR compiled binary. However, in such case, the path to the image must be provided.

It is possible to run the program with several options. The explained usage with complete list of options is as follows:

\begin{verbatim}
Usage: tabularOCR [-options] (filenames | directory name)
where options include:
    (-e | --enhance) (SIMPLE | GAMMA | EQUALIZATION)
        enhance the contrast of the image before processing
    (-g | --greyscale) (AVG | MIN | MAX | LUMA)
        set the image mode to greyscale before processing
    (-b | --binarize) (OTSU | SAUVOLA)
        binarize image before processing
    -p | --preprocess
        preprocess image with the default preprocessing options
        before processing
    -sk | --deskew
        deskew image before processing
    --output-json
        output the result in a json file
    --output-image
        output the result in an png file as a bounding box
        around each cell
\end{verbatim}
Examples of usage: 
\begin{itemize}
    \item \texttt{tabularOCR -e EQUALIZATION --binarize OTSU 11-1.jpg}
    
    This command will perform the table extraction on a binarized \texttt{11-1.jpg} image (with the exact method of binarization being Otsu) with enhanced contrast (with the exact method of contrast enhancement being histogram equalization).
    
    \item \texttt{tabularOCR ./image\_directory}
    
    This command will process all the images from\\
    \texttt{your-build-directory/bin/Debug/image\_directory}
    and save their results into\\
    \texttt{your-build-directory/bin/Debug/results/image\_directory} directory.
    
    \item \texttt{tabularOCR --output-image 11-1.jpg 7-1.jpg 5-1.jpg}
    
    This command will process the images \texttt{11-1.jpg}, \texttt{7-1.jpg}, \texttt{5-1.jpg} in the given order and save their results as \texttt{11-1.png}, \texttt{7-1.png} and \texttt{5-1.png} files, excluding JSON output.
    
\end{itemize}
    
The internal C++ structure of the project is as follows:
\begin{itemize}
    \item \texttt{main.cpp} --- used for calling the individual functions of all other files consecutively.
    \item \texttt{parser} --- contains functions for parsing the input command line arguments, file manipulation and error handling.
    \item \texttt{preprocess} --- contains the calls of individual preprocessing methods of Leptonica depending on the parser's output.
    \item \texttt{process} --- the core of the program. Contains the Tesseract API calls and the implementation of table recognition algorithm including various heuristics.
    \item \texttt{utils} --- contains several helper functions and structures used in the \texttt{process} file.
\end{itemize}